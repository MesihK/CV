%-------------------------
% Resume in Latex
% Author : Sourabh Bajaj
% License : MIT
%------------------------

\documentclass[letterpaper,11pt]{article}

\usepackage{latexsym}
\usepackage[empty]{fullpage}
\usepackage{titlesec}
\usepackage{marvosym}
\usepackage[usenames,dvipsnames]{color}
\usepackage{verbatim}
\usepackage{enumitem}
\usepackage[hidelinks]{hyperref}
\usepackage{fancyhdr}


\pagestyle{fancy}
\fancyhf{} % clear all header and footer fields
\fancyfoot{}
\renewcommand{\headrulewidth}{0pt}
\renewcommand{\footrulewidth}{0pt}

% Adjust margins
\addtolength{\oddsidemargin}{-0.5in}
\addtolength{\evensidemargin}{-0.5in}
\addtolength{\textwidth}{1in}
\addtolength{\topmargin}{-.5in}
\addtolength{\textheight}{1.0in}

\urlstyle{same}

\raggedbottom
\raggedright
\setlength{\tabcolsep}{0in}

% Sections formatting
\titleformat{\section}{
	\vspace{-4pt}\scshape\raggedright\large
}{}{0em}{}[\color{black}\titlerule \vspace{-5pt}]

%-------------------------
% Custom commands
\newcommand{\resumeItem}[2]{
\item\small{
		\textbf{#1}{: #2 \vspace{-2pt}}
	}
}

\newcommand{\resumeSubheading}[4]{
\vspace{-1pt}\item
	\begin{tabular*}{0.97\textwidth}{l@{\extracolsep{\fill}}r}
		\textbf{#1} & #2 \\
		\textit{\small#3} & \textit{\small #4} \\
	\end{tabular*}
}

\newcommand{\resumeSubItem}[2]{\resumeItem{#1}{#2}}

\renewcommand{\labelitemii}{$\circ$}

\newcommand{\resumeSubHeadingListStart}{\begin{itemize}[leftmargin=*]}
\newcommand{\resumeSubHeadingListEnd}{\end{itemize}}
\newcommand{\resumeItemListStart}{\begin{itemize}}
\newcommand{\resumeItemListEnd}{\end{itemize}}

%-------------------------------------------
%%%%%%  CV STARTS HERE  %%%%%%%%%%%%%%%%%%%%%%%%%%%%


\begin{document}

%----------HEADING-----------------
%TODO: fix speling errors
\begin{tabular*}{\textwidth}{l@{\extracolsep{\fill}}r}
	%\textbf{\href{http://web.itu.edu.tr/kilincmes/}{\Large Mesih Veysi Kılınç}} & \href{mailto:m.kilinc@gtu.edu.tr}{m.kilinc@gtu.edu.tr}\\
	\textbf{\href{http://web.itu.edu.tr/kilincmes/}{\Large Mesih Veysi Kılınç}} & \href{mailto:mesihkilinc@gmail.com}{mesihkilinc@gmail.com}\\
	\href{https://github.com/MesihK}{https://github.com/MesihK} & +90-543-864-2148 \\
	%\textit{\small Turkish Nationality - Male - Married - Age 26}
\end{tabular*} 
%\vspace{15pt}

%I am a Ph.D. Candidate in Computer Science. I do also hold a Bachelor degree in Electronics Engineering.  
%Interested in Linux Kernel and love doing embedded related works.
%Fascinated by programming and started to code during the first year of high school, doing it since then.
%I do have a fair bit of knowledge on \textit{Linux, C, C++, Python, Keras, Embedded Systems.}
\vspace{-5pt}

\section{\href{https://www.google.com/search?q=mesih+site\%3Alkml.org}{{Open Source Contributions - Linux Kernel Patchsets}}}
\resumeItemListStart
    {
	    \item \href{https://lkml.org/lkml/2018/12/2/202}{\textbf{Applied to Mainline:} initial support for "suniv" Allwinner new ARM9 SoC}
	    \vspace{-5pt}\item \href{https://lkml.org/lkml/2018/12/2/259}{Add support for DMA and audio codec of F1C100s}
	    \vspace{-5pt}\item \href{https://lkml.org/lkml/2019/2/11/131}{Timer \& SPI support for Allwinner suniv F1C100s}
    }
\resumeItemListEnd \vspace{-18pt}

%\section{Technical Skills Summary}
%\parbox{.60\linewidth}{
%	\centering
%\begin{tabular*}{0.30\textwidth}{ @{\extracolsep{\fill} } l l l l }
%	\multicolumn{2}{l}{\textbf{Programming Tools \& Languages }}\hspace{15pt} & \multicolumn{2}{l}{\textbf{Operating Systems \& Software}} \\
%	C/C++ & 5 Year  & Linux \& GNU Utils\hspace{8pt} & 3 Year \\
%	QT & 3 Year & Linux Kernel & 2 Year \\
%	GStreamer \hspace{8pt} & 2 Year & Yocto & 2 Year \\
%	Bash & 3 Year  & Kicad & 1 Year \\
%	\\
%\end{tabular*}\vspace{-30pt}
%}

%kendinden bahset, aciklama ekle. Neden linux kernel bahset
%bildigin teknolojilerden bahset. qt gstreamer kicad 

%-----------EDUCATION-----------------
\section{{Education}}
  \resumeSubHeadingListStart
    \resumeSubheading
      {Gebze Technical University}{Kocaeli, Turkey}
      {Ph.D. in \textbf{Computer Science};  GPA: -/4.00}{July 2020 -- Present}
     \vspace{-10pt}\resumeSubheading
      {Gebze Technical University}{Kocaeli, Turkey}
      {Master of Science in \textbf{Computer Science};  GPA: 3.85/4.00}{Feb. 2018 -- July 2020}
	  \resumeItemListStart\vspace{-5pt}
	\resumeItem{Courses Taken}{Deep Learning, Symbolic Computation, Robot Control Theory,
	Non-Linear Control Theory, Advanced Computer Architecture, Algorithm Analysis and Design,
	Special Topics on Algorithms}
	\resumeItem{Scientific Preparation (Undergrad Level)}{Object Oriented Programming, Data Structures,
		    Operating Systems, Computer Architecture, Discrete Mathematics} 
      \resumeItemListEnd\vspace{-7pt}
    \resumeSubheading
      {Istanbul Technical University}{Istanbul, Turkey}
      {Bachelor of Engineering in \textbf{Electronics and Communication}}{Sep. 2011 -- Feb. 2016}
  \resumeSubHeadingListEnd\vspace{-18pt}


  %-----------EXPERIENCE-----------------
\section{{Experience}}
  \resumeSubHeadingListStart

    \resumeSubheading
      {Gebze Technical University}{Kocaeli, Turkey}
      {Research Assistant}{Feb. 2018 - Present}
      %telsizden bahsedilebilri
      

	%\vspace{2pt}
	\textbf{Zenom Real-time Simulator:}
	   Improved \href{https://github.com/GTUKontrolRobotik/Zenom}{Zenom} simulation software to handle hardware targets.
	   Also developed \href{https://github.com/GTUKontrolRobotik/ZenomCore}{ZenomCore} for resource limited environments.
	   \textbf{(Qt, Non-Linear Control)}

	
	\vspace{-2pt}\textbf{Haptic Delta Robot:}
	  Designed hardware to control a delta robot and developed a software
	  to experiment different control techniques on the device.
	  \textbf{(MCU, Linear Control)}


	\vspace{-2pt}\textbf{Swarm Unmanned Aerial Vehicle:}
	Designed both hardware and software of a telemetry
	modem and an RSSI sensor capable of measuring the strength of RF signals in the specified band.
	\textbf{(RFIC, Embedded)}

	\vspace{-5pt}\resumeSubheading
      {\href{https://www.otokar.com/en-us/Pages/default.aspx}{Otokar}}{Sakarya, Turkey}
      {Software Engineer}{August 2016 - Feb. 2018}
      %simulation tool'dan bahset
      %hsm'ye gecisi anlat
      
      
	%\vspace{2pt}
	\textbf{Border Surveillance and Reconnaissance Vehicle:}
	At one of our software when we try to implement a new
	feature, old ones started to broke. We had tremendous time pressure.
	I stepped back and decided to change the whole core code.
	Introduced a hierarchical state machine that manages the device.
	Since states managed correctly, code size shrunk and started to behave correctly.
	Later on, I revised all other devices used in this vehicle platform to use state machines.
	(\textbf{Embedded, C++, Qt, Yocto})


	\vspace{-4pt}\textbf{Programmable Socket Simulator:}
	Because we used actual hardware it was very hard to try all code changes to make sure we didn’t break anything.
	Wrote a simulator to understand the devices in the vehicle better and fix bugs.
	The Result was faster development and hunting long-lasting bugs. 
	(\textbf{Python})
	

	\vspace{-5pt}\resumeSubheading
      {\href{https://www.ctech.com.tr/}{CTech}}{Istanbul, Turkey}
      {Software Engineer}{July 2015 - August 2016}
      %muxer hikayesini anlat
      %linux driveri yazdigin zamandan bahset


	%\vspace{2pt}
	\textbf{UAV Modem:}
	I met Linux with this project. Hardware was given to me to write a device driver. Surprised to
	find that the driver already existed, and found that with a little tweaking it just started to work.
	The more surprising thing was how powerful the shell was.
	Without even writing a line of code it was so easy to connect little things to create a powerful program that utilizes my hardware.
	Later I wrote a device driver for a custom FPGA block.
	(\textbf{Linux Kernel, Device Drivers})

	  
	\vspace{-5pt}\textbf{Linux Training:}
	Took Linux training from Nazim Koc (\href{http://ucanlinux.com}{uCanLinux.com}). (\textbf{U-Boot, Linux Kernel, Busybox compilation and building RootFS})



  \resumeSubHeadingListEnd\vspace{-10pt}

  %-----------PROJECTS-----------------
\vspace{-5pt}\section{Major Projects}
  \resumeSubHeadingListStart

%\item
%\textbf{Real-time Video Compression Device:}
%Coded software for an H264 video compression device.
%It is based on IMX6 SoC and also uses real-time patched Linux and GStreamer.
%Build the required device tree and userspace software that is based on GStreamer
%\textbf{(Gstreamer, Yocto)}

\item
\textbf{Voice Transceiver Device:} An embedded Linux system with Opus codec and Alsa. 
Designed the hardware, and coded necessary \textbf{ALSA SoC driver, SoC's DMA Driver, RFIC device driver} and userspace software.
Build the device tree and RootFS.
This was the project where I created several patchsets and sent them to the mainline Linux Kernel.
\textbf{(Linux Driver Development, Cross-compiling, Porting)}
  \resumeSubHeadingListEnd%\vspace{-10pt}

  %-------------------------------------------
\end{document}
